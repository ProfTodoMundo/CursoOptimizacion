% !TEX root = NotasCursoOptimizacion.tex

%===========================================
\section{Fundamentos}
%===========================================

\begin{Def}[Valores y vectores propios]
Sea $A$ una matrix de $n\times n$, $v$ vector de dimensi\'on $n$ y $\lambda$ escalar. Se dice que $v$ es un vector propio de $A$ y $\lambda$ es vector propio de $A$ si se cumple
\begin{eqnarray*}
Av=\lambda v
\end{eqnarray*}
\end{Def}

\begin{Note}
De la igualdad anterior se tiene $Av-\lambda v=\left(A-I\lambda\right)v=0$, ecuaci\'on que siempre tiene como  soluci\'on $v=0$, para cualquier $\lambda$. Si $\left(A-I\lambda\right)$ es invertible, la \'unica soluci\'on es $v=0$. Para tener soluciones no triviales, se requiere que $\lambda$ sea tal que  $\left(A-I\lambda\right)$ no sea invertible, entonces
\begin{eqnarray*}
det\left(A-I\lambda\right)=0
\end{eqnarray*}
\end{Note}

\begin{Teo}
Sea $A$ matriz de $n\times n$, $\lambda$ es un valor propio de $A$ si y s\'olo si
\begin{eqnarray*}
p\left(\lambda\right)=det\left(A-I\lambda\right)=0
\end{eqnarray*}
\end{Teo}

\begin{Def}
La ecuaci\'ion $p\left(\lambda\right)=0$ se denomina polinomio caracter\'istico de $A$.
\end{Def}


\begin{Teo}
Sea $A$ una matriz de $n\times n$, y sean $\lambda_{1},\lambda_{2},\dots,\lambda_{m}$ valores caracter\'isticos de $A$ con vectores propios $v_{1},v_{2},\dots,v_{m}$. Entonces los vectores propios correspondientes a valores propios distintos son linealmente independientes.
\end{Teo}

\begin{Def}
Sea $\lambda$ valor propio de $A$, la multiplicidad geom\'etrica de $\lambda$ es la dimensi\'on del espacio propio correspondiente a $\lambda$.
\end{Def}


\begin{Propty}[Matrices]


\end{Propty}


\begin{Def}Semejantes
Sean $A$ y $B$ matrices de $n\times n$ son semejantes si existe una matriz invertible $C$ de $n\times n$ tal que
\begin{eqnarray}
B=C^{-1}AC.
\end{eqnarray}
\end{Def}


\begin{Teo}


\end{Teo}

\begin{Def}[Diagonalizacion]
Una matriz $A$ de $n\times n$ es diagonalizable si existe una matriz diagonal $D$ tal que $A$ es semejante a $D$.
\end{Def}


\begin{Teo}Diagonalizacion
Una matriz $A$ de $n\times n$ es diagonalizable si y s\'olo si tiene $n$ valores propios linealmente independientes, en este caso la matriz diagonal $D$ tiene como elementos en la diagona los $n$ valores propios de $A$. Entonces $D=C^{-1}AC$.
\end{Teo}

\begin{Teo}
Sea $A$ matriz sim\'etrica real de $n\times n$ , entonces $A$ tiene $n$ vectores propios reales ortonormales.
\end{Teo}

\begin{Def}
Se dice que $A$ matriz de $n\times n$ es diagonalizable ortogonalmente si existe una matriz ortogonal $Q$ tal que $Q^{t}AQ=D$, donde $D=diag\left(\lambda_{1},\lambda{2},\dots,\lambda_{n}\right)$ son los valores propios de $A$.
\end{Def}

\begin{Teo}
Sea $A$ matriz real de $n\times n$, entonces $A$ es diagonalizable ortogonalmente si y s\'olo si $A$ es sim\'etrica.
\end{Teo}


\begin{Def} Formas Cuadr\'aticas
\begin{itemize}
\item[i) ] Una ecuaci\'on cuadr\'atica en dos variables sin t\'erminos lineales es una ecuaci\'on de la forma $ax^{2}+bxy+cy^{2}=d$, donde $|a|+|b|+|c|\neq0$.

\item[ii) ] Una forma cuadr\'atica en dos variables es una expresi\'on de la forma $F\left(x,y\right)=ax^{2}+bxy+cy^{2}$, donde $|a|+|b|+|c|\neq0$.
\end{itemize}
\end{Def}


\begin{Def} Sobre formas cuadr\'aticas
Sea $A$ matriz sim\'etrica, entonces se define la forma cuadr\'atica $F\left(x,y\right)=Av\cdot v$.
\end{Def}

\begin{Ejem}
Si $F\left(x,y\right)=ax^{2}+bxy+cy^{2}$ es forma cuadr\'atica, sea $A=\left(\begin{array}{cc}
a & b/2\\
b/2 & a\\
\end{array}
\right)$
entonces $Av\cdot v=d$.
\end{Ejem}

\begin{Def}
Sea $v=\left(\begin{array}{c}
x_{1}\\
x_{2}\\
\vdots\\
x_{n}\\
\end{array}
\right)$ y sea $A$ matriz sim\'etrica de $n\times n$. Entonces una forma cuadr\'atica en $x_{1},x_{2},\dots,x_{n}$ es una expresi\'on de la forma $F\left(x_{1},x_{2},\dots,x_{n}\right)=Av\cdots v$.
\end{Def}


\begin{Teo}Matrices positivas definidas I


\end{Teo}


\begin{Teo}Matrices positivas definidas II


\end{Teo}

\begin{Def} Segmento

\end{Def}



\begin{Def} Convexo

\end{Def}


\begin{Def} Funciones convexas

\end{Def}

\begin{Teo} Sobre funciones convexas


\end{Teo}


\begin{Def} Funciones cuasiconvexas

\end{Def}


\begin{Propty} Sobre funciones convexas


\end{Propty}

\begin{Propty} Sobre funciones cuasiconvexas


\end{Propty}

%===========================================
\section{Introducci\'on}
%===========================================

\begin{Def} Optimizacion sin restricciones


\end{Def}

